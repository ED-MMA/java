\documentclass{scrreprt}
\usepackage{hyperref}
\usepackage{listings}
\usepackage{underscore}
\usepackage{graphicx}
\usepackage[utf8]{inputenc}
\usepackage[english]{babel}
\usepackage{enumitem}
\usepackage{svg}
\usepackage{amsmath}
\usepackage{biblatex}
\usepackage{color}
\usepackage{comment}

\definecolor{dkgreen}{rgb}{0,0.6,0}
\definecolor{gray}{rgb}{0.5,0.5,0.5}
\definecolor{mauve}{rgb}{0.58,0,0.82}

\setlist[enumerate]{label=\arabic*.}
\setlist[enumerate,2]{label*=.\arabic*}

\hypersetup{
    bookmarks=false,    % show bookmarks bar?
    pdftitle={Software Requirement Specification},    % title
    pdfauthor={Papp Levente László JANEQB},                     % author
    pdfsubject={Software specification for EDMMA},                        % subject of the document
    pdfkeywords={EDMMA, SPEC, SOFTWARE}, % list of keywords
    colorlinks=true,       % false: boxed links; true: colored links
    linkcolor=blue,       % color of internal links
    citecolor=black,       % color of links to bibliography
    filecolor=black,        % color of file links
    urlcolor=purple,        % color of external links
    linktoc=page            % only page is linked
}%
\def\myversion{1.0 }
\date{}
%\title
\begin{document}

    \begin{flushright}
        \rule{16cm}{5pt}\vskip1cm
        \begin{bfseries}
            \Huge{SOFTWARE\\ Specifikáció}\\
            \vspace{1.5cm}
            a\\
            \vspace{1.5cm}
            Programozás alapjai 3.\\
            \vspace{1.5cm}
            tárgyhoz\\
            \vspace{1.5cm}
            \LARGE{Verzió \myversion}\\
            \vspace{1.5cm}
            Készítette : Papp Levente (JANEQB) (176584@bme.hu)\\

            \vspace{1.5cm}
            \today\\
        \end{bfseries}
    \end{flushright}

    \tableofcontents



    \chapter{Bevezetés}\label{ch:bevezetes}

    Az \href{https://www.elitedangerous.com/}{Elite: Dangerous\texttrademark} játékban a végtelen univerzum végtelen lehetőséget biztosít.
    Lehetsz hauler, trader, outlaw vagy éppen zsoldos is.
    Ez a program a zsoldosnak állt embereket hivatott segíteni.

    \section{A zsoldosokról}\label{sec:a-zsoldosokrol}
    A zsoldos létnek egy kulcsfontosságú eleme a megfelelő csillag cluster kiválasztása.
    Vannak, úgynevezett kalóz frakciók, melyek egy adott csillagrendszerben ha dominánsak, akkor a körülötte lévő rendszerekben a nem-kalóz frakciók zsoldosokat keresnek a kiírtásukra.
    Vagy legalábbis a megcsonkításukra.

    \section{Alap küldetés mechanika}\label{sec:alap-kuldetes-mechanika}
    A játékban egy adott játékos egyszerre maximum 20 küldetést vihet magával.
    A küldetések teljesítése soros módon történik, időrend beli sorrendben.
    Azonban a játék lehetőséget ad ezeknek a küldetéseknek a párhuzamos teljesítésére.

    \subsection{Küldetés mechanika példa}\label{subsec:kuldetes-mechanika-pelda}
    A legegyszerűbb eset:
    \begin{enumerate}[label=\thechapter.\arabic*.]
        \item Van 1 frakció A, aki a játékosnak 3 küldetést adott, összesen 80 kalózt kellene kiiktatni.
        \item A játékos elmegy a cél csillagrendszerbe és kiírt 80 kalózt.
        \item A játékos visszatér a frakcióhoz és leadja a küldetéseket.
    \end{enumerate}
    A párhuzamos teljesítés esete:
    \begin{enumerate}[label=\thechapter.\arabic*.]
        \item Van 3 frakció A, B, C, akik a játékosnak 3 küldetést adtak, összesen 60 kalózt kellene kiiktatni.
        \item A játékos elmegy a cél csillagrendszerbe és kiírt 20 kalózt.
        Mivel a játékos 3 különböző frakciótól vette fel a küldetéseket, ezért a játék lehetőséget ad arra, hogy a küldetések párhuzamosan teljesüljenek.
        \item A játékos visszatér a frakciókhoz és leadja a küldetéseket.
    \end{enumerate}

    \chapter{Rendszeren belüli axiómák}\label{ch:rendszeren-beluli-axiomak}
    \begin{enumerate}[label=\thechapter.\arabic*.]
        \item Minden log file jsonl formátumú
        \item Minden json object megfeleltethető az event-nek amit a játékos végzett
        \item Az eventek időrendi sorrendben íródnak a log file-ba
        \item Minden log file az alábbi könyvtárban található:
        \subitem \textbf{Windows:} \\
        userhome\textbackslash Saved Games\textbackslash Frontier Developments\textbackslash Elite Dangerous
        \subitem \textbf{Linux:}  \\
        userhome/.steam/steam/SteamApps/compatdata/359320/pfx/drive\_c/users/steamuser\\
        /Saved Games/Frontier Developments/Elite Dangerous
        \item A további információk a log file-okról az alábbi linken találhatóak: \\\url{https://elite-journal.readthedocs.io/en/latest/File\%20Format/#file-location}
    \end{enumerate}

    \chapter{A program részei}\label{ch:a-program-reszei}
    \begin{enumerate}[label=\thechapter.\arabic*.]
        \item A program 2 részből áll:
        \subitem \textbf{1.} A log file-ok feldolgozásából, melyeket a játék generál.
        \subitem \textbf{2.} A feldolgozott adatok alapján egy statisztika generálásából.
        \item A program a feldolgozott adatokat egy adatbázisban tárolja.
        \item A program a feldolgozott adatokat egy grafikus felületen jeleníti meg.
    \end{enumerate}

\end{document}